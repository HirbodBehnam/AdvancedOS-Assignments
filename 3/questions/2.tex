\section{\lr{Small Files}}
در ابتدا برنامه‌ای که می‌نویسیم بدین صورت است که 1000 فایل با طول اسم بین ۵ تا ۱۰ را در فولدری می‌سازند.
این فولدر برای تست من در
\lr{HDD}
است. در قسمت
\lr{write}
با توجه به تجربایت سوال قبل مجبور شدم که اندازه‌ی هر فایل برابر
\lr{4096}
بایت یا اندازه‌ی یک
\lr{page}
باشد.

با این احتساب برنامه را اجرا می‌کنیم و نتایج را در زیر می‌آوریم:

\smalltitle{\lr{Cached}:}
\samplebox{real    0m0.115s\\
user    0m0.095s\\
sys     0m0.020s} % Write cache
\smalltitle{\lr{Direct}:}
\samplebox{real    0m24.643s\\
user    0m0.445s\\
sys     0m0.154s} % Write direct

همان طور که مشخص است سرعت برنامه با
\lr{cache}
بسیار بیشتر از برنامه عادی است که به دلیل این است که دیتا در مموری نوشته می‌شود و بعدا به صورت
\lr{periodic}
در دیسک نوشته می‌شود. اما از طرفی در
\lr{direct IO}
مطمئن می‌شویم که دیتا در دیسک نوشته شده است. لازم به ذکر است که بعد از هر ران فایل‌های فولدر مذکور را کامل پاک می‌کردم
که خود فایل‌سیستم سربار اضافه تری اضافه نکند بر روی آن.

حال سناریو
\lr{read}
را طراحی می‌کنیم. در این سناریو فولدر داده شده را در ابتدا می‌خوانیم و لیست تمام فایل‌های آن را در می‌آوریم.
سپس هر کدام از آنها را باز می‌کنیم و می‌خوانیم. از آنجا که هر کدام از فایل‌ها به اندازه‌ی یک
\lr{page}
هستند پس صرفا یک
\lr{read syscall}
کافی است. همچنین قبل از اجرای این دستورات با دستورات سوال اول کش را پاک می‌کنیم. نتایج بنچمارک به صورت زیر هستند:


\smalltitle{\lr{Cached}:}
\samplebox{real    0m1.174s\\
user    0m0.006s\\
sys     0m0.095s} % Read cache
\smalltitle{\lr{Direct}:}
\samplebox{real    0m1.470s\\
user    0m0.002s\\
sys     0m0.073s} % Read direct

در اینجا نیز همان طور که مشخص است با
\lr{cache}
سرعت بیشتر می‌شود ولی این کمی عجیب است! چرا که کلا هر بار یک
\lr{page}
می‌خوانیم و اصلا خود فایل بیشتر نمی‌تواند
\lr{cache}
شود. پس منطقا باید سرعت
\lr{cache} و \lr{direct}
با هم برابر می‌شد! این موضوع به نظرم نشان دهنده‌ی این است که یا مثلا
\lr{file tree}
نیز
\lr{cache}
می‌شود یا اینکه لینوکس سعی می‌کند که فایل‌های بعدی را همزمان باز کند. مثلا وقتی ببیند که داریم فایل‌های
یک فولدر را به ترتیب می‌خوانیم حدس بزند که از اینجا به بعد نیز می‌خواهیم همین کار
را بکنیم و جلو جلو فایل‌ها را باز کند و کش کند.