\section{\lr{Page Cache}}
به نظر من بهتر است که در ابتدا خارج از برنامه فایل مذکور را ساخته و سپس بعد برنامه‌ای بنویسیم که فقط خواندن و سپس
فقط نوشتن را انجام دهد. برای بیشتر حس شدن تاثیر
\lr{read} و \lr{write}
من با اینکه سیستم خودم بر روی
\lr{nVME}
نصب است ولی بر روی یک هارد دیسک دیتا را ذخیره می‌کنم که کند باشد. این پارتیشن با فایل سیستم
\lr{ext4}
فرمت شده است.

برای ساختن فایل از دستور زیر استفاده می‌کنیم:
\samplebox{dd if=/dev/urandom of=/media/hirbod/LinuxHDD/sample.dat count=10240 bs=1M}

\samplebox{real    2m48.880s\\
user    0m1.650s\\
sys     0m28.315s} % Read Cache

\samplebox{real    26m16.402s\\
user    0m10.612s\\
sys     2m41.620s} % Read Direct

\samplebox{real    30m6.216s\\
user    0m3.986s\\
sys     4m11.283s} % Write Cache