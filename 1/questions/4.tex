\smalltitle{سوال 4}
\\\noindent
من زمانی که بر روی اندازه‌گیری
\lr{TLB shootdown}
به کمک
\lr{perf}
سرچ می‌کردم، به
\link{https://www.jabperf.com/how-to-deter-or-disarm-tlb-shootdowns/}{این}
لینک رسیدم که تقریبا خواسته‌ی سوال بود! با این حال بدون کپی کردن کد و از خودم کد را نوشتم.

خود دستور
\lr{madvise}
به سیستم‌عامل
\lr{hint}هایی
جهت استفاده از مموری داده شده به عنوان پارامتر را می‌دهد. در این کد من از
\lr{MADV\_DONTNEED}
استفاده کردم. با توجه به
\link{https://man7.org/linux/man-pages/man2/madvise.2.html}{\lr{man page}}
این تابع می‌توان نتیجه گرفت که به صورت خلاصه این
\lr{hint}
به سیستم‌عامل می‌گوید که این قسمت از مموری در آینده‌ی نزدیک استفاده نمی‌شود.
این باعث می‌شود که یک
\lr{TLB shootdown}
برای این
\lr{page}
از مموری اتفاق بیفتد که جا در
\lr{TLB}
آزاد شود.

در این قسمت نیز برای
\lr{align}
کردن مموری به
\lr{page}ها
از
\lr{pvalloc}
استفاده کردم.

برای اجرای
perf
نیز از دستورات زیر استفاده کردم:
\samplebox{perf stat -e itlb.itlb\_flush,tlb\_flush.dtlb\_thread,tlb\_flush.stlb\_any,page-faults,cache-misses ./3.out 1}

حال به سوالات پرسیده شده جواب می‌دهیم.
\begin{enumerate}
    \item نتایج بدست آمده از اسکریپت در زیر قابل مشاهده است.
    \begin{latin}
    \centering
    \begin{tabular}{ccccccc}
    \hline
    \csvreader[no head,table head=\hline,late after line=\\\hline]{codes/3-stat.csv}{1=\one,2=\two,3=\three,4=\four,5=\five,6=\six,7=\seven}
    {\one & \two & \three & \four & \five & \six & \seven}
    \end{tabular}
    \end{latin}
    همان طور که مشخص است تمامی انواع
    \lr{TLB shootdown}ها
    با بالا رفتن تعداد ترد‌ها زیاد می‌شوند. دلیل بالا رفتن تعداد
    \lr{TLB shootdown}ها
    همان طور که در قسمت قبل گفته شد، خالی کردن
    \lr{TLB}
    برای
    \lr{page}های
    دیگر است.
    نکته‌ای که چشم من را گرفت این است که
    \lr{page fault}
    با اینکه به صورت کاملا صعودی بالا رفته است،‌ ولی تعداد آن خیلی زیاد نیست. ما در هر ترد صد هزار بار
    \lr{madvise}
    می‌کنیم. اما برای هر تردی که اضافه می‌شود تعداد 
    \lr{page fault}ها
    در حدود ۲ تا زیاد می‌شود. این رفتار به نظر من تقصیر
    \lr{madvise}
    نیست. بلکه احتمالا تقصیر ساختن ترد‌ها و یا طول کشیدن بیشتر برنامه است. دقت کنید که مموری
    \lr{allocate}
    شده نیز برابر
    ۳۲ کیلوبایت یا
    ۸ صفحه است.
    همچنین مشاهده می‌شود که
    \lr{cache miss}
    نیز زیاد می‌شود. این موضوع می‌تواند دو دلیل داشته باشد. یا اینکه صرفا مثل
    \lr{page fault}
    برای ساخت ترد باشد، یا اینکه دستور
    \lr{madvise}
    باعث می‌شود که مموری داده شده از
    \lr{cache}
    نیز پاک شود.

    همچنین مشخص است که هر چه قدر تعداد
    \lr{TLB shootdown}های
    ما بیشتر باشد برنامه نیز زمان بیشتری اجرایش طول می‌کشد.
    \item نمودار‌های خواسته شده را بر اساس جدول بالا رسم می‌کنیم.
    \begin{figure}[H]
        \centering
        \begin{tikzpicture}
        \begin{axis}
        \addplot table [x=threads, y=itlb, col sep=comma] {codes/3-stat.csv};
        \end{axis}
        \end{tikzpicture}
        \caption{\lr{ITLB shootdown} بر حسب تعداد ترد}
    \end{figure}
    \begin{figure}[H]
        \centering
        \begin{tikzpicture}
        \begin{axis}
        \addplot table [x=threads, y=dtlb, col sep=comma] {codes/3-stat.csv};
        \end{axis}
        \end{tikzpicture}
        \caption{\lr{DTLB shootdown} بر حسب تعداد ترد}
    \end{figure}
    \begin{figure}[H]
        \centering
        \begin{tikzpicture}
        \begin{axis}
        \addplot table [x=threads, y=stlb, col sep=comma] {codes/3-stat.csv};
        \end{axis}
        \end{tikzpicture}
        \caption{\lr{STLB shootdown} بر حسب تعداد ترد}
    \end{figure}
    \begin{figure}[H]
        \centering
        \begin{tikzpicture}
        \begin{axis}
        \addplot table [x=threads, y=pagefault, col sep=comma] {codes/3-stat.csv};
        \end{axis}
        \end{tikzpicture}
        \caption{\lr{Page fault} بر حسب تعداد ترد}
    \end{figure}
    \begin{figure}[H]
        \centering
        \begin{tikzpicture}
        \begin{axis}
        \addplot table [x=threads, y=cachemiss, col sep=comma] {codes/3-stat.csv};
        \end{axis}
        \end{tikzpicture}
        \caption{\lr{Cache miss} بر حسب تعداد ترد}
    \end{figure}
    \item \phantom{text}
    \begin{figure}[H]
        \centering
        \begin{tikzpicture}
        \begin{axis}
        \addplot table [x=threads, y=time, col sep=comma] {codes/3-stat.csv};
        \end{axis}
        \end{tikzpicture}
        \caption{زمان اجرا به ثانیه بر حسب تعداد ترد}
    \end{figure}
\end{enumerate}