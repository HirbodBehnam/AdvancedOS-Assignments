\section*{مقدمه}
در ابتدا من می‌خواهم که کمی درباره‌ی پیکربندی ماشین‌های مجازی که توسط خود شما به ما داده شده بود توضیح دهم.
در ابتدا کل
\lr{package}های
سیستم را آپدیت کردم. در ابتدا فقط
\lr{mirror}های
ماشین مجازی را بر روی سرور‌های ایران تنظیم کردم که سرعت دانلود بیشتر باشد. بعد از نصب آن‌ها ماشین‌های مجازی را
ری استارت کردم که ورژن جدید کرنلی که دانلود شده بود استفاده شود.
(در جفت ماشین‌ها ورژن کرنل \lr{5.4.0-147} است.)
سپس برنامه‌ی
\lr{stress}
را به کمک
\lr{apt}
نصب کردم. در ادامه برای نصب
\lr{perf}
نیز از دستور زیر استفاده کردم:
(\link{https://askubuntu.com/a/578618/746382}{منبع})
\samplebox{sudo apt install linux-tools-common linux-tools-generic linux-tools-`uname -r`}
در ادامه نیز به کمک دستور
\lr{hostname}،
\lr{hostname}
هر ماشین مجازی را به
\lr{hirbod-behnam-99171333-vm1} و \lr{hirbod-behnam-99171333-vm2}
تغییر دادم که صرفا هم اسمم مشخص باشد و هم شماره‌ی ماشین مجازی.

در نهایت نیز به کمک دستورات زیر کمی درباره‌ی سخت افزار هر یک از سیستم‌ها اطلاعات کسب کردم.
\begin{latin}
\begin{itemize}
    \item \codeword{lscpu}: 3 CPU Cores, Intel(R) Xeon(R) CPU E5-2699 v4 @ 2.20GHz
    \item \codeword{free -h}: 9GB RAM and 4 GB swap space
    \item \codeword{df -h}: 50 GB total disk size, 40 GB free
    \item \codeword{lsblk -d -o name,rota}: All disks are HDD
    \item \codeword{systemd-detect-virt}: VMWare hypervisor
    \item \codeword{cat /etc/os-release}: Ubuntu 20.04.6 LTS
\end{itemize}
\end{latin}
همچنین به کمک نرم افزار
\lr{htop} یا \lr{btop}
نیز می‌توان میزان مصرف منابع سیستم را دید. برای راحتی نیز از
\lr{SSH key}
برای
\lr{login}
کردن استفاده می‌کنم.