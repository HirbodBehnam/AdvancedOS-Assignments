\section{msync}
در ابتدا کد نوشته شده را شرح می‌دهم. در ابتدای کد فایل یک مگابایتی در کنار فایل اجرایی می‌سازیم.
محتوای این فایل صرفا کارکتر
\codeword{NULL}
است. سپس این فایل را به کمک دستور
\codeword{mmap}
به صورت
\lr{memory mapped}
می‌خوانیم. سپس به تعداد آرگومان‌های برنامه ترد ایجاد می‌کنیم و روی هر ترد یک بایت از فایل را عوض
می‌کنیم و با دستور
\codeword{msync}
دیتا را بلافاصله در دیسک ذخیره می‌کنیم. در قسمت اول این سوال از
\codeword{MS\_SYNC}
استفاده می‌کنیم. این فلگ به تابع
\codeword{msync}
می‌گوید که تا وقتی که تغییرات در دیسک ذخیره نشده است، از تابع بر نگرد.
در نهایت نیز صبر می‌کنیم که کار ترد‌ها تمام شود و فایل مورد نظر را
\lr{unmap}
می‌کنیم. همچنین دقت کنید که هر ترد بر روی یک هسته‌ی به خصوص اجرا می‌شوند.
(مثل قسمت و تمرین قبل)
همچنین با اینکه در تمرین نوشته شده بود که از 2 ترد استفاده کنید، ولی من فکر کردم که شاید اجرای
برنامه با تعداد ترد متغیر بتواند اطلاعات بیشتری به ما بدهد.

کد
C
و کد‌های
\lr{bash}
برای درست کردن
\lr{csv}های
مربوط به تمرین و دیتای‌خام در فولدر
\lr{codes}
موجود است.

اطلاعات مربوط به اجرای برنامه در جداول
\ref{table:msync_bare_metal} و \ref{table:msync_vm1} و \ref{table:msync_vm2}
آمده است.

\begin{figure}[H]
    \begin{latin}
        \centering
        \begin{tabular}{ccccc}
        \hline
        \csvreader[no head,table head=\hline,late after line=\\\hline]{codes/2-bare-metal.csv}{1=\one,2=\two,3=\three,4=\four,5=\five}
        {\one & \two & \three & \four & \five}
        \end{tabular}
    \end{latin}
    \caption{جدول اطلاعات مربوط به اجرای برنامه بر روی سیستم \lr{bare metal}}
    \label{table:msync_bare_metal}
\end{figure}
\begin{figure}[H]
    \begin{latin}
        \centering
        \begin{tabular}{ccccc}
        \hline
        \csvreader[no head,table head=\hline,late after line=\\\hline]{codes/2-vm1.csv}{1=\one,2=\two,3=\three,4=\four,5=\five}
        {\one & \two & \three & \four & \five}
        \end{tabular}
    \end{latin}
    \caption{جدول اطلاعات مربوط به اجرای برنامه بر روی ماشین مجازی اول}
    \label{table:msync_vm1}
\end{figure}
\begin{figure}[H]
    \begin{latin}
        \centering
        \begin{tabular}{ccccc}
        \hline
        \csvreader[no head,table head=\hline,late after line=\\\hline]{codes/2-vm2.csv}{1=\one,2=\two,3=\three,4=\four,5=\five}
        {\one & \two & \three & \four & \five}
        \end{tabular}
    \end{latin}
    \caption{جدول اطلاعات مربوط به اجرای برنامه بر روی ماشین مجازی دوم}
    \label{table:msync_vm2}
\end{figure}

دقت کنید که همان طور که گفته شد به دلیل مجازی سازی نمی‌توان تعداد
\lr{tlb-miss}ها
را پیدا کرد بر روی یک ماشین مجازی.

حال به تحلیل داده‌ها می‌پردازیم. در ابتدا سعی می‌کنیم که تحلیل کنیم که با افزایش تعداد ترد‌ها چه اتفاقی
می‌افتد. با افزایش تعداد ترد‌ها تعداد
\lr{TLB shootdown}ها
به صورت خیلی آهسته بیشتر می‌شود. این بیشتر شدن تعداد
\lr{TLB shootdown}ها
به قدری کم است که حتی من می‌توانم آنرا تقصیر
\lr{pthreads}
بیندازم! همین حرف را نیز می‌توان برای
\lr{page fault}
زد. با زیاد شدن تعداد ترد‌ها نیز می‌توان مشاهده کرد که تعداد
\lr{TLB miss}ها
به شدت و تا حدودی خطی/نمایی افزایش پیدا می‌کند.
نکته‌ی جالبی که به نظر من وجود دارد این است که تعداد
\lr{TLB shootdown}ها
در حد چند عدد افزایش پیدا کرده است ولی تعداد
\lr{TLB miss}ها
در حدود میلیون‌تا افزایش پیدا کرده است. این به ما نشان می‌دهد که دلیل
\lr{miss}ها
این نیست که دیتا از
\lr{TLB}
پاک می‌شود که منشا آن
\lr{TLB shootdown}
است. بلکه ممکن است که یا به دلیل این باشد که واقعا دیتا‌هایی از صفحات خیلی زیادی داریم می‌خوانیم.
یا اینکه به دلیلی
\lr{TLB}
ما پر می‌شود و مجبور می‌شویم که کلی از داده‌هایمان را از آن بیرون بیندازیم. اما به نظر من این دو
مورد خیلی نمی‌توانند خوب توجیح کنند که چرا با
\emph{افزایش ترد‌ها}
\lr{TLB miss}
نیز افزایش پیدا می‌کند. تنها دلیلی که شاید بتوانم به آن فکر کنم شاید این باشد که وقتی که ترد‌ها را
پشت سر هم اجرا می‌کنیم، شاید مثلا ترد‌ها به فاصله‌ی بیشتر از یک
\lr{memory page}
بر روی
\lr{memory mapped file}
کار می‌کنند طوری که مجبور می‌شوند جای همدیگر را در
\lr{TLB}
بگیرند.
به عبارتی یکی دیتای دیگری را بیرون می‌اندازد و برعکس.
اما به نظرم باز هم این به قدری نیست که در حد میلیونی تعداد
\lr{TLB miss}ها زیاد شود
و به صورت کلی به نظرم احتمال این اتفاق کم است.

 من برای کمی آزمایش بیشتر حجم فایل را نیز به ده مگابایت تغییر دادم که صرفا چک کنم که چه متغیر‌هایی عوض
می‌شوند. نتیجه بر روی کامپیوتر شخصی من در جدول
\ref{table:msync_10mb}
آمده‌ است.
\begin{figure}[H]
    \begin{latin}
        \centering
        \begin{tabular}{ccccc}
        \hline
        \csvreader[no head,table head=\hline,late after line=\\\hline]{codes/2-10mb.csv}{1=\one,2=\two,3=\three,4=\four,5=\five}
        {\one & \two & \three & \four & \five}
        \end{tabular}
    \end{latin}
    \caption{اطلاعات مربوط به اجرای برنامه با حجم فایل ده مگابایت}
    \label{table:msync_10mb}
\end{figure}

متوجه می‌شویم که
\lr{TLB shootdown}ها
زیاد نشده است. پس واقعا ممکن است که این
\lr{shootdown}ها
تقصیر ساخت و بستن ترد‌ها باشد! از طرفی
\lr{page fault}
زیاد شد. این اتفاق به احتمال خوبی برای این می‌افتد که لینوکس از
\lr{demand paging}
برای
\codeword{mmap}
استفاده می‌کند. تعداد
\lr{TLB miss}ها
نیز بسار زیاد شد. این نیز صرفا به خاطر این می‌تواند رخ داده باشد که آن
\lr{miss}ها
در نوشتن و
\lr{derefrence}
کردن آدرس مموری مربوط به فایل
\lr{memory mapped}
رخ می‌دهد. چرا که تقریبا تمامی اعداد ما ۱۰ برابر شدند.

حال سعی می‌کنیم ماشین‌های مجازی را با ماشین
\lr{bare metal}
مقایسه کنیم. با توجه به مقایسه اعداد می‌توان نتیجه گرفت که این دو تقریبا هیچ فرقی ندارند
و چه مجازی شده شده باشند و چه نشده باشند و چه چندین ماشین با هم بر روی یک
\lr{host}
بخواهند از دستور
\codeword{msync}
استفاده کنند، مشکل سرعت به خاطر
\lr{TLB shootdown}های
زیاد پیش نمی‌آید. ولی دقت کنید که دیسک فوق العاده زیادی درگیر می‌شود که بلافاصله دیتا در دیسک
نوشته شود.

حال قسمت امتیازی تمرین را می‌زنیم. کافی است که پرچم تابع
\codeword{msync}
را
\codeword{MS\_ASYNC}
بکنیم. آزمایش را دوباره انجام می‌دهیم و نتیجه‌ها را در جداول
\ref{table:msync_async_bare_metal} و \ref{table:msync_async_vm1} و \ref{table:msync_async_vm2}
می‌توان دید.

\begin{figure}[H]
    \begin{latin}
        \centering
        \begin{tabular}{ccccc}
        \hline
        \csvreader[no head,table head=\hline,late after line=\\\hline]{codes/2-async-bare-metal.csv}{1=\one,2=\two,3=\three,4=\four,5=\five}
        {\one & \two & \three & \four & \five}
        \end{tabular}
    \end{latin}
    \caption{جدول اطلاعات مربوط به اجرای برنامه \lr{async} بر روی سیستم \lr{bare metal}}
    \label{table:msync_async_bare_metal}
\end{figure}
\begin{figure}[H]
    \begin{latin}
        \centering
        \begin{tabular}{ccccc}
        \hline
        \csvreader[no head,table head=\hline,late after line=\\\hline]{codes/2-async-vm1.csv}{1=\one,2=\two,3=\three,4=\four,5=\five}
        {\one & \two & \three & \four & \five}
        \end{tabular}
    \end{latin}
    \caption{جدول اطلاعات مربوط به اجرای برنامه \lr{async} بر روی ماشین مجازی اول}
    \label{table:msync_async_vm1}
\end{figure}
\begin{figure}[H]
    \begin{latin}
        \centering
        \begin{tabular}{ccccc}
        \hline
        \csvreader[no head,table head=\hline,late after line=\\\hline]{codes/2-async-vm2.csv}{1=\one,2=\two,3=\three,4=\four,5=\five}
        {\one & \two & \three & \four & \five}
        \end{tabular}
    \end{latin}
    \caption{جدول اطلاعات مربوط به اجرای برنامه \lr{async} بر روی ماشین مجازی دوم}
    \label{table:msync_async_vm2}
\end{figure}

در این قسمت نیز مشاهده می‌شود که اعداد ما خیلی فرقی با اعداد جداول
\ref{table:msync_bare_metal} و \ref{table:msync_vm1} و \ref{table:msync_vm2}
ندارند. پس می‌توان گفت که
\lr{async}
بودن یا نبودن تاثیری بر روی
\lr{TLB shootdown} و \lr{TLB miss} و \lr{page fault}
و حتی زمان اجرای برنامه ندارد.